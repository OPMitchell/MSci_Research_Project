\documentclass{sig-alternate-05-2015}


\begin{document}

% Copyright
\setcopyright{acmcopyright}
%\setcopyright{acmlicensed}
%\setcopyright{rightsretained}
%\setcopyright{usgov}
%\setcopyright{usgovmixed}
%\setcopyright{cagov}
%\setcopyright{cagovmixed}

% DOI
\doi{xx.xxx/xxx_x}

% ISBN
\isbn{xxx-xxxx-xx-xxx/xx/xx}

%Conference
\conferenceinfo{}{}

\acmPrice{}

%
% --- Author Metadata here ---

%\CopyrightYear{2007} % Allows default copyright year (20XX) to be over-ridden - IF NEED BE.
%\crdata{0-12345-67-8/90/01}  % Allows default copyright data (0-89791-88-6/97/05) to be over-ridden - IF NEED BE.
% --- End of Author Metadata ---

\title{A System for Simulation-based, Emergent, Interactive Narrative Generation}
\subtitle{\large\textrm{INTERIM REPORT\\ (16/11/2018)}}
%
% You need the command \numberofauthors to handle the 'placement
% and alignment' of the authors beneath the title.
%
% For aesthetic reasons, we recommend 'three authors at a time'
% i.e. three 'name/affiliation blocks' be placed beneath the title.
%
% NOTE: You are NOT restricted in how many 'rows' of
% "name/affiliations" may appear. We just ask that you restrict
% the number of 'columns' to three.
%
% Because of the available 'opening page real-estate'
% we ask you to refrain from putting more than six authors
% (two rows with three columns) beneath the article title.
% More than six makes the first-page appear very cluttered indeed.
%
% Use the \alignauthor commands to handle the names
% and affiliations for an 'aesthetic maximum' of six authors.
% Add names, affiliations, addresses for
% the seventh etc. author(s) as the argument for the
% \additionalauthors command.
% These 'additional authors' will be output/set for you
% without further effort on your part as the last section in
% the body of your article BEFORE References or any Appendices.

\numberofauthors{2} %  in this sample file, there are a *total*
% of EIGHT authors. SIX appear on the 'first-page' (for formatting
% reasons) and the remaining two appear in the \additionalauthors section.
%
\author{
% You can go ahead and credit any number of authors here,
% e.g. one 'row of three' or two rows (consisting of one row of three
% and a second row of one, two or three).
%
% The command \alignauthor (no curly braces needed) should
% precede each author name, affiliation/snail-mail address and
% e-mail address. Additionally, tag each line of
% affiliation/address with \affaddr, and tag the
% e-mail address with \email.
%
% 1st. author
\alignauthor
Oliver Mitchell\\
       \affaddr{University of Nottingham}\\
       \email{psyom1@nottingham.ac.uk}
% 2nd. author
\alignauthor
Peter Blanchfield\\
       \affaddr{University of Nottingham}\\
       \email{peter.blanchfield@nottingham.ac.uk}
}
% There's nothing stopping you putting the seventh, eighth, etc.
% author on the opening page (as the 'third row') but we ask,
% for aesthetic reasons that you place these 'additional authors'
% in the \additional authors block, viz.

\maketitle
\begin{abstract}
Recent research in the field of interactive storytelling (IS) has focused on the advantages of ``emergent narratives'' (ENs). With this approach, the storyline ``emerges'' at run-time from the interactions between author-programmed autonomous agents and the player. The benefit is a non-linear narrative, allowing for greater player interactivity. However, ENs are unpredictable at authoring time, removing the author's control over the plot structure. The struggle to balance the author's control over game narrative whilst simulatenously allowing the player to influence the story's direction has been dubbed the ``Narrative Paradox''. We discuss the design and implementation of a new hybrid emergent storytelling system for creating unscripted narratives which aims to integrate the successful components of existing approaches, including an appraisal-driven agent architecture extended to model autobiographical memory, and a ``virtual director'' which maintains storyworld knowledge and uses a novel metaheuristic genetic algorithm to generate suitable global events for agents to appraise and respond to. An evaluation of the system's effectiveness (based on comparative user-testing) is discussed with a particular regard to player ``immersion'' and believability of the generated narrative.
\end{abstract}

% We no longer use \terms command
%\terms{Theory}

\keywords{interactive storytelling; games; 
autonomous virtual agents; emergent narratives}

\section{Background}

\subsection{Defining Interactive Storytelling}

Interactive storytelling is a unique form of narrative design seen in digital entertainment mediums where no pre-determined storyline exists. Instead, the user can change the story and influence the characters themselves. Bostan and Marsh define interactive storytelling as an approach to computer game narrative design where ``real-time feedback collected from the players is used by the game engine to continuously modify the content as it is being delivered.'' [1]. The key aspect of interactive storytelling is that the player is an active participant and their in-game actions have a direct effect on the narrative that is told. This differs from traditional computer game narratives which typically consist of an unchanging linear story with a pre-defined outcome.

\subsection{Problems with the Integration of Interactivity into Game Narratives}

In most computer games, attempts at complex, dynamic storylines are usually ``confined to simple branching tree structures'' [3]. Furthermore, as the scope of games increases and development effort is focused on assets such as textures, 3D-modelling and other gameplay content, a greater number of narrative possibilities arise. However, the ``static nature of these [tree] structures affects the range of options covered [and] reduces the framework for player interaction''. Louchart et al. suggest ``the integration of interactivity into game narratives requires a rethinking of the design process'' [3] because these tree-like structures either expand exponentially with the inclusion of more narrative elements (such as additional characters added to the game) or have to remain so simple and rigid that the player's immersion in the game world is threatened.

In short, with a greater amount of gameplay content comes infeasible exponential growth in the number of branches of a narrative tree structure. If the amount of potential story branches is too high, developers are forced to simplify the story, resulting in less (or no) player interactivity.

\subsection{The Narrative Paradox}

It is common for narrative design in computer games to follow a plot-based approach, meaning that complete control over every aspect of the story is required by the author, leaving little room for interaction by the player. Louchart and Aylett recognised that the ``authoring of interactive narrative presents the paradox that ... the author requires control over the unfolding narrative whilst ... the user expects freedom over their decisions.'' [3][4]. These two goals clearly conflict. The greater the level of interactivity provided to the player, the less control the author has over the story's events. Equally, the more control the author has over the story, the less room there is for the player to interact with characters and change the course of the narrative. This concept has been dubbed the ``Narrative Paradox'' [5]. Louchart and Aylett suggest that a solution to this paradox should divert from the branching plot-based approaches implemented in current computer games, hence also solving the problem of an exponentially increasing number of diverging storylines.

\subsection{Emergent Narrative}

Aylett proposed Emergent Narrative (EN) as an alternative. EN is an approach to designing interactive narratives where, instead of a fixed, authored plot, the``narrative weight of the application is shared by authors and players.'' [3][5][6]. This is typically achieved by simulating a virtual world, including virtual characters with different behaviours created by the author. The storyline then `emerges' at run-time from the interactions between these autonomous agents, and the player [2]. The advantage of an EN approach is a non-linear narrative that allows for a greater level of player interactivity. Despite this, ENs are unpredictable at authoring time. Kriegel and Aylett explain how authors are forced to ``let go of specific story lines altogether and ... focus on creating the elements from which the story will emerge.'' [7]. ENs work as a sort of `sandbox' where the author defines the limits of the story world and creates the characters and behaviours that fill it, then leaves the simulation to create its own story, providing no further input during run-time.\\

\noindent ENs benefit from having no rigid structure to their story because there is no branching tree restricting the range of potential storylines. This solves the problem of infeasible exponential growth and allows for a potentially limitless number of possible storylines (should the simulation be complex enough). However, the narrative paradox remains. In fact, emergent narratives are the opposite of the plot-based approach; where plot-based approaches require high levels of authorship and direction, restricting player interactivity, emergent narratives implicitly prevent the author from giving any direction at all, making them highly unpredictable.

Yet another disadvantage of emergent narrative approaches are their reliance on well implemented virtual agents and their control logic. If the AI controlling each agent is too simple, then the story is unlikely to be very compelling or complex enough to satisfy the player. Or, if the AI has too limited a set of possible responses to stimuli in its environment, the player may see the same storylines again and again, threatening their immersion in the narrative.

\subsection{Motivation}

\noindent The motivation for this project is to develop a new approach to interactive storytelling which utilises and combines components of successful, existing emergent narrative systems in order to generate compelling narratives and give players a greater sense of agency, whilst simulatenously providing adequate control of the narrative to the author. This paper focuses on simulation-based approaches which exploit the usefulness of emergent narratives, and omit the restrictions of plot-based narrative designs.

\section{Related Work}
\subsection{History of Digital Storytelling}
\subsubsection{Early approaches}
Since the 1970s, there have been many digital storytelling systems developed both in research and for commercial games. Early examples of story generation include \textit{Novel Writer}, a system that generated 2000+ word murder mystery stories that emerged from the actions simulated by each character [8], and \textit{TaleSpin}, a complex system that models rational behaviour for each character, including the production of personal goals that each agent can attempt to achieve autonomously through the creation of its own plan; a series of actions it may take [9]. \textit{Virtual Storyteller} was another system that, despite a lack of interaction, created stories via the actions and cooperation of intelligent agents. In this system, plots are not pre-defined but created by the actions of the agents alongside a ``virtual director''; a seperate agent that maintains general story-world knowledge about the plot structure which it uses to judge whether a character's intended actions fit into the narrative in order to produce a ``good'' plot. Consequently, unlike the full-autonomy displayed in \textit{TaleSpin}, the agents in \textit{Virtual Storyteller} can be considered only semi-autonomous because they are consistently guided in their actions to create a well-structured plot [10]. 

Despite producing compelling emergent narratives, none of these systems can be considered examples of interactive storytelling because they do not allow a user to change the path of the story at run-time (\textit{Novel Writer} and \textit{TaleSpin} only accepted user input for the initial storyworld state and \textit{Virtual Storyteller} allows for no user input whatsoever, relying solely on its encapsulated virtual director to direct the plot).\\

\subsubsection{Towards Emergent Narratives}

\noindent Following these first approaches came a greater focus on emergent storytelling and a move towards non-branching, character-focused systems. Aylett first coined the term ``emergent narrative'' in 1999 when discussing the narrative issues arising from placing children's TV characters in a virtual environment where they and the user have a joint spacial existence [6]. The original television narrative was wholly pre-scripted and viewed passively by the audience from camera positions determined by the creators. This strategy is unsuitable for use in a virtual environment where the user is transformed from a spectator to a potential participant. Aylett argues that a pre-determined narrative means the role of the user must also be pre-determined. Therefore, relaxing this constraint allows the user to be more freely involved.

The solution is to build the narrative from the bottom-up through the interactions of the characters. Many groups have looked at producing emergent narratives by treating characters as virtual actors [barbara hayes roght ``virtual theater''] [IMPROV virtual actors system 97]. In such a scenario, the focus of the user is moved from a participant to a director, where they're able to construct narratives involving virtual actors and themselves. This has been repeatedly compared to the playtime of young children where they switch repeatedly between particpant and director in order to control the direction of their roleplay [6][children reference]. 

\subsection{Existing Emergent Narrative Systems}

\subsubsection{Automated Planning with HTNs}

Cavazza et al. produced a prototype character-based system which focused on defining the possible actions of individual characters in hierarchical task networks (HTN) in order to create automated plans and achieve character desired goals [11]. Characters act autonomously, executing available actions in their HTN in order to reach their goal, similar to the \textit{TaleSpin} system [9].  Each action is categorised and used to determine other characters' reactions (e.g. characters will react negatively to ``rude'' actions). The system also includes ``mood values'' for each character which can affect other characters' plans. Because the system allows for emergent behaviour, unexpected situations may arise that are unaccounted for in a character's plans. This creates the need for ``situated reasoning'' and ``action repair''. Cavazza et al. explains that situated reasoning ``originates from the discrepancy between an agent's expectations and action preconditions'' and is focused on the character aiming to achieve a specific result in a given situation, or avoiding an undesirable result. Typically, this is seen when a character's current plan is interrupted and they must deal with a specific unanticipated situation. Action repair is focused on recovering from action failure, when external factors threaten the satisfaction of executability conditions. This system is primarily plan focused and authoring constitutes describing the sub-tasks in each character's hierarchical task network, and the character's reactions to ``various generic situations, mostly arising
from the conjunction of actions from the characters'
respective plans'' [11].\\ 

\subsubsection{Modelling Social Interaction: PromWeek}

\newline McCoy et al. produced \textit{Prom Week} [12], a game described as a ``social physics'' simulator which utilises a bespoke AI system: \textit{Comme il Faut} (CiF) [13]. The goal of the project was to design a game that could provide ``satisfying stories that reflect player choices in a wide possibility space'' [12].
As the title suggests, \textit{Prom Week} revolves around the social lives of a group of teenagers the week before their prom night. The player is given a set of goals which they must complete within the week, for example; finding a date to attend the prom with. McCoy et al.'s goal was to decrease the volume of explicitly authored story space whilst increasing the amount explorable by the player. They reasoned the best approach was to build ``a social artificial intelligence (AI) system that computationally models social space and social interaction'', instead of having the author explicitly detail each possible interaction (a burdensome and effectively intractable task should the number of interactions be large enough). With their proposed AI system (CiF),  the author provides ``reusable and recombinable representations of social norms and social interactions'' (or social rules) for autonomous agents in the model [13]. The social state existing across these agents is modified by multi-character social interactions, or social exchanges that utilise these representations. In effect, autonomous agents (the characters) perceive the global social state and attempt to alter it to accomplish their own social goals. This is achieved through social interactions, all of which follow the author-specified rules.\\

For example, if the player has the goal of attaining a date to attend the prom with, they may try to use the ``flirt'' social interaction on another character. This may, or may not, succeed and the social state across both characters will be altered according to the author-specified rules that dictate the social norm of that interaction. The results of this interaction are stored in a database of social facts, acting as a sort of global memory. CiF also incorporates `trigger rules' which ``encode the cascading effects of social state change'' outside of the current interaction [13]. Once an interaction has finished, these triggers are run over the social network in order to update the social state of each character.

\newline The CiF system itself does not attempt to model the entire story world. It is a social reasoning component that aims to realistically model the rules and patterns that character should follow during any social exchange. In contrast to Cavazza et al.'s hierarchical task network planning approach [11] (and other approaches such as behaviour trees), CiF encapsulates domain knowledge in one large rulebase that represents the social norms of a given story world. The system chooses each character's behaviour based on these rules. CiF doesn't create a static series of events, but the logic of the story world, the characters involved, and their goals. CiF is inherently driven by the simulation of social exchanges, therefore the way in which goals are met is entirely emergent and unplanned. This support Aylett's initial description of emergent narratives as emerging from the interactions of characters, rather than being pre-planned [5][6].

\newline An extension of the CiF system [14] was proposed by Guimaraes et al. that adapted the social physics simulator to a first-person perspective, implemented as a publicly downloadable modification in the \textit{Elder Scrolls V: Skyrim} game engine. The result was an extended interaction space for in-game NPCs leading to more believable characters and an improved player experience, as confirmed by the modification's popularity [15] and the results of user testing.

\subsubsection{Modelling Emotional Behaviour - FearNot!}
Louchart et al. produced The \textit{FearNot!} (Fun with Empathic Agents Reaching Novel Outcomes in Teaching!) storytelling system, designed for anti-bullying education and inspired by Forum Theatre [3]. \textit{FearNot!} generates short episodes where a character is bullied and the victim asks the player for advice. The player's advice has an impact on the victim's emotional state which in turn influences the actions of that character in the next episode. Each character is represented by an intelligent agent architecture which includes an ``affective appraisal system and autonomous action selection capabilities, producing an emergent narrative'' [3]. In addition to the character agents, an additional story facilitator agent is used for selecting characters, props, and locations for successive episodes. The story facilitator takes the player's actions and the characters' emotional state into consideration, generating an unscripted episode which is logically coherent to previously transpired events. For example, if the player ``has advised the character to hit back, it may set up an episode where victim and bully confront each other directly''[3]. The story facilitator is comparable to \textit{Virtual Storyteller}'s ``virtual director'' [10] which stores global story data and uses it to influence and adapt the narrative.

\textit{FearNot!} takes a focus on character-based emergent narratives which Louchart et al. explain requires a "bottom-up approach in which story elements are synthesised in real time via character interaction". This contrasts branching plot-based narratives which are created top-down, the story comprised of fixed plot elements. With \textit{FearNot!'s} approach, characters are defined by their emotions, personalities, action tendencies, goals and emotional reactions. This is "organic" because characters will only perform actions in-character, according to their own personality and desires, rather than requiring global action management. From the author's point of view, developing a character requires the static checking of character actions (e.g. making sure all actions have reachable preconditions) and the simulation of interaction in various contexts.

\textit{FearNot!} uses an affective agent architecture dubbed FAtiMA (\textbf{F}earnot \textbf{A}ffec\textbf{TI}ve \textbf{M}ind \textbf{A}rchitecture) [19][20]. The architecture is designed to use author-specified emotions and personality to influence the behaviour of characters. It includes two essential components: a continuous planner (allowing characters to act intelligently and perform action repair and situated reasoning during unexpected events) and an emotional personality model (which effects the way character react and provides a greater sense of believability). Appraisal is the assessment of any given event in relation to the character's emotions and influences the character's choice of actions towards their goals (world states they desire). FAtiMA allows for the description of appraisal rules which define the event to be appraised and the desirability of the event on the target character [3]. Using this data, the character's emotions are updated. For example, an event in which a character is bullied would have very low desirability and produce negative emotions for the victim character.

Emotions are a crucial aspect of the agent architecture and influence each character's planning process. Emotional values can be used as pre-conditions for potential action. The use of emotional values (amongst other data, such as memories and social relations) can be used by agents to perform rational processing to intelligently select actions that will help them towards their goal. However, FAtiMA also incorporates non-cognitive, reactional processing: "spontaneous reactions triggered by intense emotions [and] not part of the planning process" [3]. With this feature, characters with extreme emotional states can react with appropriate actions; for example, a character with extremely negative emotions may burst out crying, spontaneously. This is an attempt to allow characters to react wholly emotionally to certain situations rather than just cognitively and adds an extra layer of believability.

\subsection{Emergent Narrative in Commercial Games}
Emergent narrative generation methods are not solely practiced in research; many commercially successful games have yielded interesting and unique methods for generating unplanned interactive storylines:

\subsubsection{The Sims}
\textit{The Sims} [16] is possibly one of the most widely played and easily recognisable video games of the simulation genre and is a textbook example of emergent narrative creation. 
From the outset the player is provided with great freedom, being able to create their own character; namely their appearance, but later installments in the franchise allow for the selection of personality traits that help define unique characters and change the results of social interactions. There is no rigid storyline to the game and no plot the player must follow. Instead, the user ``plays god'' and instructs their avatar to perform desired actions such as socialising with other characters or interacting with objects in the world. Typically, a narrative emerges from the results of the player's actions and the interactions of the player's avatar with other characters (e.g. neighbours or family members). For example, the player may choose to use the ``flirt'' social interaction with another character they meet in a bar who will respond accordingly. This could lead to an in-game relationship, marriage, and even starting a family.

\textit{The Sims} includes deep simulations of personality and social traits, not dissimilar to \textit{PromWeek}'s social AI system. Characters can also have immediate responses to emotional state, alike the \textit{FearNot!} system's action tendencies. For example, if a character's mood is low enough they will ignore the player's actions in order to satisfy their own needs and improve their mood first.

Despite the high level of freedom given to the player, a serious limitation of \textit{The Sims}' narrative generation is the fact that the quality of the story relies entirely on the player's choice of interesting actions. Should they desire, the player could allow their avatar to live in complete isolation and only watch TV all day, thus resulting in a lack of narrative entirely. There is no guarantee that engaging high-quality narratives will be produced because the player acts autocratically as spectator and director.

\subsubsection{RimWorld}
\textit{RimWorld} [17] is a ``colony simulator'' game that was inspired by the highly realistic simulation-based approach to game design pioneered by \textit{Dwarf Fortress} [18], a sandbox game which sees the player manage the livelihood of a group of dwarves.

\textit{RimWorld} tasks the player with managing a group of spaceship crash survivors as they construct a colony on their new home: an alien planet. The player must ``manage colonists' moods, needs, individual wounds, and illnesses'' [17]. The game is driven by an intelligent AI storyteller and its creator describes the game not as a strategy game, but as a story generator in its own right. At any given moment, the AI storyteller analyzes the colony's situation and decides which event will make the best story [17]. Accordingly, the AI storyteller can be considered \textit{RimWorld}'s equivalent of \textit{Virtual Storyteller}'s ``virtual director'' [10], or \textit{FearNot!'s} story facilitator;  a seperate agent that maintains story-world knowledge about the plot structure and uses it to adapt the narrative at run-time. As a result, \textit{RimWorld} never suffers from the narrative limitations experienced in \textit{The Sims} where the quality of the story relies wholly on how compelling a player's actions are. Because the AI storyteller acts as an agent itself, its goal can be considered the generation of a ``good'' story and its process of selecting suitable actions to change the narrative's course can be considered a form of action repair, ensuring that player actions that divert from its goal are eventually remedied. 

For example, players have the choice of upgrading the colony with weapons and defences to protect characters from periodic waves of attackers. The AI storyteller can evaluate the colony's level of defense and ensure that the next wave of attackers is strong enough to prove challenging to the player and therefore exciting enough to produce a ``good'' narrative.

\textit{RimWorld} includes an advanced social simulation inspired by \textit{The Sims}. Colonists have traits which help them react in unique ways to social interactions (e.g. ``Night-Owl'' colonists have a better mood if they're up at night, ``Greedy'' colonists have a mood loss if their bedrooms aren't impressive, and ``Misandrist'' colonists have an automatically lower opinion of male characters), and emotions which change according to various stimuli. Similar to the \textit{FearNot!} system, the state of these emotions determine which actions a character will take. Colonists can also reactively process emotions, identical to \textit{FearNot!}'s action tendencies. This manifests as a ``mental break'' where extreme mood values cause colonists to disregard player instructions entirely and engage in fully autonomous, usually obstructive, behaviour. For example, if a colonist's mood is low enough they may engage in an uncontrollable psychotic episode of pyromancy, igniting the colony's buildings until their mood values return to normal.

Colonists have a basic memory too and are able to remember the effects of actions for predetermined lengths of time. For example, a recently divorced colonist may have a mood reduction for several months of in-game time. However, memories of actions do not  influence the decision making process as colonists act according to their present mood values and relationship values with other colonists.

\section{Designing a Hybrid System}
\subsection{Criteria}
\subsection{Agent Architecture}
\subsubsection{Emotional Personality Model}
\subsubsection{Continuous Planning}
\subsection{Goals}
\subsubsection{Active Pursuit vs. Interest}}
\subsubsection{Managing and Executing Actions}
\subsubsection{Action Priority Queueing}
\subsubsection{Need for Asynchronicity}
\subsection{Virtual Director}
\subsubsection{Potential for Genetic Algorithm}

\section{Implementation}
\subsection{Unity}
\subsection{Prototype System}

\section{Project Progress}
\subsection{Project Management}
\subsection{Contributions and Reflections}
\subsection{Moving Forward}

\newpage
\addcontentsline{toc}{section}{References}

\begin{thebibliography}{20}

\bibitem{Bostan and Marsh}
Bostan B., \& Marsh T. (2012). 
``Fundamentals Of Interactive Storytelling''. 
\textit{Academic Journal of Information Technology}, \textbf{3}(8), pp. 20-42

\bibitem{Riedl}
Riedl M. O. (2010). 
``A Comparison of Interactive Narrative System Approaches using Human Improvisational Actors''.
\textit{Georgia Institute of Technology, College of Computing}

\bibitem{Authoring Emergent Games}
Louchart S., Aylett R., Kriegel M., Dias J., Figueiredo R., \& Paiva A. (2008).
``Authoring Emergent Narrative-Based Games''.
\textit{Journal of Game Development}, \textbf{3}(1), pp. 19-37

\bibitem{Solving the Narrative Paradox in VEs}
Louchart S., \& Aylett R. (2003).
``Solving the Narrative Paradox in VEs - Lessons from RPGs''.
\textit{In: Rist T., Aylett R.S., Ballin D., Rickel J. (eds) Intelligent Virtual Agents. IVA 2003. Lecture Notes in Computer Science, Vol. 2792}, pp. 244-248

\bibitem{EmergentNarrative}
Aylett R. (2000).
``Emergent Narrative, Social Immersion and ``Storification''''.
\textit{Proceedings Narrative and Learning Environments Conference}, Edinburgh, Scotland

\bibitem{Towards Emergent Narrative}
Aylett R. (1999).
``Narrative in Virtual Environment - Towards Emergent Narrative''.
\textit{In: Proceedings of the AAAI Fall Symposium on Narrative Intelligence, 1999}

\bibitem{Massively Collaborative Authoring}
Kriegel M., \& Aylett R. (2008).
``Emergent Narrative As A Novel Framework For Massively Collaborative Authoring''.
\textit{In: Prendinger H., Lester J., Ishizuka M. (eds) Intelligent Virtual Agents. IVA 2008. Lecture Notes in Computer Science, vol 5208}

\bibitem{Novel Writer}
Klein S., Aeschlimann J.F., Balsiger D.F., Converse S.L., Court C., Foster M., Lao R., Oakley J.D., \& Smith J. (1973).
``Automatic Novel Writing: A Status Report''.
\textit{University of Wisconsin}

\bibitem{Tale-Spin}
Meehan, J.R. (1977).
``Tale-Spin, an Interactive Program that Writes Stories''.
\textit{In: Proceedings of the Fifth International Joint Conference on Artificial Intelligence (IJCAI)}

\bibitem{The Virtual Storyteller}
Theune, M., Faas S., Nijholt, A., \& Heylen, D. (2003).
``The Virtual Storyteller: Story Creation by Intelligent Agents''.
\textit{In: Proceedings of TIDSE 2003: Technologies for Interactive Digital Storytelling and Entertainment}

\bibitem{Character-based Interactive Storytelling}
Cavazza, M., Charles, F., \& Mead, S.J. (2002).
``Character-based Interactive Storytelling''.
\textit{IEEE Intelligent Systems}, \textbf{17}(4), pp.17-24

\bibitem{PromWeek}
McCoy, J., Treanor, M., Samuel, B., Reed, A., Mateas, M., \& Wardrip-Fruin, N. (2013).
``Prom Week: Designing past the game/story dilemma''.
\textit{In: Proceedings of the Eighth International Conference on the Foundations of Digital Games (FDG 2013)}, pp.94-101

\bibitem{CiF}
McCoy, J., Treanor, M., Samuel, B., Wardrip-Fruin, N., \& Mateas, M (2013).
``Comme il Faut: A System for Authoring Playable Social Models''.
\textit{In: Proceedings of the Seventh AAAI Conference on Artificial Intelligence and Interactive Digital Entertainment}, pp.158-163

\bibitem{CiF-CK}
Guimaraes, M., Santos, P., \& Jhala, A. (2017).
``CiF-CK: An Architecture for Social NPCs in Commercial Games''.
\textit{2017 IEEE Conference on Computational Intelligence and Games (CIG)}, pp.126-123

\bibitem{CiF-CK Mod Page}
Social NPCs Mod for Elder Scrolls V: Skyrim, available at: https://www.nexusmods.com/skyrim/mods/77792/ [Accessed 21 Nov. 2018]

\bibitem{The Sims}
Maxis: The Sims. Electronic Arts (2000)

\bibitem{RimWorld}
Tynan Sylvester: RimWorld. Ludeon Studios (2018), available at: https://rimworldgame.com/ [Accessed 21 Nov. 2018]

\bibitem{Dwarf Fortress}
Adams, T., Adams, Z.: Slaves to Armok: God of Blood Chapter II: Dwarf Fortress (2006)

\bibitem{FAtiMA}
Dias, J., Macscarenhas, S., \& Paiva, A.
``FAtiMA Modular: Towards an Agent Architecture with
a Generic Appraisal Framework''.
\textit{Emotion Modeling} pp.44-56

\bibitem{Emotional Agents}
Dias, J., \& Paiva, A.
``Feeling and Reasoning: A Computational Model for Emotional Agents''.
\textit{In Proceedings of 12th Portuguese Conference on Artificial Intelligence, EPIA 2005,} pp.127-140


\end{thebibliography}
 
\end{document}
