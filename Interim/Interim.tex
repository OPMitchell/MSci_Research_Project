\documentclass{sig-alternate-05-2015}


\begin{document}

% Copyright
\setcopyright{acmcopyright}
%\setcopyright{acmlicensed}
%\setcopyright{rightsretained}
%\setcopyright{usgov}
%\setcopyright{usgovmixed}
%\setcopyright{cagov}
%\setcopyright{cagovmixed}

% DOI
\doi{xx.xxx/xxx_x}

% ISBN
\isbn{xxx-xxxx-xx-xxx/xx/xx}

%Conference
\conferenceinfo{}{}

\acmPrice{}

%
% --- Author Metadata here ---

%\CopyrightYear{2007} % Allows default copyright year (20XX) to be over-ridden - IF NEED BE.
%\crdata{0-12345-67-8/90/01}  % Allows default copyright data (0-89791-88-6/97/05) to be over-ridden - IF NEED BE.
% --- End of Author Metadata ---

\title{A System for Simulation-based, Emergent, Interactive Narrative Generation}
%\subtitle{[Extended Abstract]}
%
% You need the command \numberofauthors to handle the 'placement
% and alignment' of the authors beneath the title.
%
% For aesthetic reasons, we recommend 'three authors at a time'
% i.e. three 'name/affiliation blocks' be placed beneath the title.
%
% NOTE: You are NOT restricted in how many 'rows' of
% "name/affiliations" may appear. We just ask that you restrict
% the number of 'columns' to three.
%
% Because of the available 'opening page real-estate'
% we ask you to refrain from putting more than six authors
% (two rows with three columns) beneath the article title.
% More than six makes the first-page appear very cluttered indeed.
%
% Use the \alignauthor commands to handle the names
% and affiliations for an 'aesthetic maximum' of six authors.
% Add names, affiliations, addresses for
% the seventh etc. author(s) as the argument for the
% \additionalauthors command.
% These 'additional authors' will be output/set for you
% without further effort on your part as the last section in
% the body of your article BEFORE References or any Appendices.

\numberofauthors{2} %  in this sample file, there are a *total*
% of EIGHT authors. SIX appear on the 'first-page' (for formatting
% reasons) and the remaining two appear in the \additionalauthors section.
%
\author{
% You can go ahead and credit any number of authors here,
% e.g. one 'row of three' or two rows (consisting of one row of three
% and a second row of one, two or three).
%
% The command \alignauthor (no curly braces needed) should
% precede each author name, affiliation/snail-mail address and
% e-mail address. Additionally, tag each line of
% affiliation/address with \affaddr, and tag the
% e-mail address with \email.
%
% 1st. author
\alignauthor
Oliver Mitchell\\
       \affaddr{Universiy of Nottingham}\\
       \email{psyom1@nottingham.ac.uk}
% 2nd. author
\alignauthor
Peter Blanchfield\\
       \affaddr{University of Nottingham}\\
       \email{peter.blanchfield@nottingham.ac.uk}
}
% There's nothing stopping you putting the seventh, eighth, etc.
% author on the opening page (as the 'third row') but we ask,
% for aesthetic reasons that you place these 'additional authors'
% in the \additional authors block, viz.

\maketitle
\begin{abstract}
Recent research in the field of interactive storytelling (IS) has focused on the advantages of "emergent narratives" (ENs). With this approach, the storyline 'emerges' at run-time from the interactions between author-programmed autonomous agents and the player. The benefit is a non-linear narrative, allowing for greater player interactivity. However, ENs are unpredictable at authoring time, removing the author's control over the plot structure. The struggle to balance the author's control over game narrative whilst simulatenously allowing the player to influence the story's direction has been dubbed the "Narrative Paradox". We discuss the design and implementation of a new hybrid emergent storytelling system for creating unscripted narratives which aims to integrate the successful components of existing approaches, including an appraisal-driven agent architecture extended to model autobiographical memory, and a "virtual director" which maintains storyworld knowledge and uses a novel metaheuristic genetic algorithm to generate suitable global events for agents to appraise and respond to. An evaluation of the system's effectiveness (based on comparative user-testing) is discussed with a particular regard to player "immersion" and believability of the generated narrative.
\end{abstract}

% We no longer use \terms command
%\terms{Theory}

\keywords{interactive storytelling; emergent narratives; games; autonomous virtual agents}

\section{Introduction}
Most modern computer games fail to provide a dynamic and changing narrative to their players. Instead, they confine storylines to simple branching tree structures which grow exponentially with additional game content, and leave little room for player interaction. Research in the field of interactive storytelling (IS) has focused on the advantages of "emergent narratives" (ENs). With this approach, the storyline "emerges" at run-time from the interactions between author-programmed autonomous agents and the player. The benefit is a non-linear narrative, allowing for greater player interactivity. However, ENs are unpredictable at authoring time, removing the author's control over the story's events.\\

\noindent The struggle to balance the author's control over game narrative whilst simulatenously allowing the player to influence the story's direction has been dubbed the "Narrative Paradox" [5]. This paper proposes a new system which aims to utilise and combine components of successful, existing emergent narrative systems.

\section{Background}

\subsection{Defining Interactive Storytelling}

Interactive storytelling is a unique form of narrative design seen in digital entertainment mediums where no pre-determined storyline exists. Instead, the user can change the story and influence the characters themselves. Bostan and Marsh define interactive storytelling as an approach to computer game narrative design where "real-time feedback collected from the players is used by the game engine to continuously modify the content as it is being delivered." [1]. The key aspect of interactive storytelling is that the player is an active participant and their in-game actions have a direct effect on the narrative that is told. This differs from traditional computer game narratives which typically consist of an unchanging linear story with a pre-defined outcome.

\subsection{Problems with the Integration of Interactivity into Game Narratives}

In most computer games, attempts at complex, dynamic storylines are usually ``confined to simple branching tree structures'' [3]. Furthermore, as the scope of games increases and development effort is focused on assets such as textures, 3D-modelling and other gameplay content, a greater number of narrative possibilities arise. However, the ``static nature of these [tree] structures affects the range of options covered [and] reduces the framework for player interaction''. Louchart et al. suggest``the integration of interactivity into game narratives requires a rethinking of the design process'' because these tree-like structures either expand exponentially with the inclusion of more narrative elements (such as additional characters added to the game) or have to remain so simple and rigid that the player's immersion in the game world is threatened.

In short, with a greater amount of gameplay content comes infeasible exponential growth in the number of branches of a narrative tree structure. If the amount of potential story branches is too high, developers are forced to simplify the story, resulting in less (or no) player interactivity.

\subsection{The Narrative Paradox}

It is common for narrative design in computer games to follow a plot-based approach, meaning that complete control over every aspect of the story is required by the author, leaving little room for interaction by the player. Louchart and Aylett recognised that the ``authoring of interactive narrative presents the paradox that ... the author requires control over the unfolding narrative whilst ... the user expects freedom over their decisions.'' [3][4]. These two goals clearly conflict. The greater the level of interactivity provided to the player, the less control the author has over the story's events. Equally, the more control the author has over the story, the less room there is for the player to interact with characters and change the course of the narrative. This concept has been dubbed the "Narrative Paradox" [5]. Louchart and Aylett suggest that a solution to this paradox should divert from the branching plot-based approaches implemented in current computer games, hence also solving the problem of an exponentially increasing number of diverging storylines.

\subsection{Emergent Narrative}

Aylett proposed Emergent Narrative (EN) as an alternative. EN is an approach to designing interactive narratives where, instead of a fixed, authored plot, the``narrative weight of the application is shared by authors and players.'' [3][5][6]. This is typically achieved by simulating a virtual world, including virtual characters with different behaviours created by the author. The storyline then `emerges' at run-time from the interactions between these autonomous agents, and the player [2]. The advantage of an EN approach is a non-linear narrative that allows for a greater level of player interactivity. Despite this, ENs are unpredictable at authoring time. Kriegel and Aylett explain how authors are forced to ``let go of specific story lines altogether and ... focus on creating the elements from which the story will emerge.'' [7]. ENs work as a sort of `sandbox' where the author defines the limits of the story world and creates the characters and behaviours that fill it, then leaves the simulation to create its own story, providing no further input during run-time.\\

\noindent ENs benefit from having no rigid structure to their story because there is no branching tree restricting the range of potential storylines. This solves the problem of infeasible exponential growth and allows for a potentially limitless number of possible storylines (should the simulation be complex enough). However, the narrative paradox remains. In fact, emergent narratives are the opposite of the plot-based approach; where plot-based approaches require high levels of authorship and direction, restricting player interactivity, emergent narratives implicitly prevent the author from giving any direction at all, making them highly unpredictable.

Yet another disadvantage of emergent narrative approaches are their reliance on well implemented virtual agents and their control logic. If the AI controlling each agent is too simple, then the story is unlikely to be very compelling or complex enough to satisfy the player. Or, if the AI has too limited a set of possible responses to stimuli in its environment, the player may see the same storylines again and again, threatening their immersion in the narrative.

\subsection{Motivation}

\noindent The motivation for this project is to develop a new approach to interactive storytelling that partially mitigates the problems of the narrative paradox and improves upon the drawbacks of existing solutions. This project includes the development of a prototype IS system which aims to generate compelling narratives and give players a greater sense of agency, whilst simulatenously providing adequate control of the narrative to the author. This paper focuses on simulation-based approaches which exploit the usefulness of emergent narratives, and omit the restrictions of plot-based narrative designs.

\section{Related Work}
\subsection{History of Digital Storytelling}
Since the 1970s, there have been many emergent narrative systems developed both in research and for commercial games. Early examples of story generation include \textit{Novel Writer}, a system that generated 2000+ word murder mystery stories that emerged from the actions simulated by each character [8], and \textit{TailSpin}, a complex system that models rational behaviour for each character, including the production of personal goals that each agent can attempt to achieve autonomously through the creation of its own plan; a series of actions it may take [9]. \textit{Virtual Storyteller} was another system that, despite a lack of interaction, created stories via the actions and cooperation of intelligent agents. In this system, plots are not pre-defined but created by the actions of the agents alongside a "virtual director"; a seperate agent that maintains general story-world knowledge about the plot structure which it uses to judge whether a character's intended actions fits into the narrative in order to produce a "good" plot. Consequently, unlike the full-autonomy displayed in \textit{TailSpin}, the agents in \textit{Virtual Storyteller} can be considered only semi-autonomous because they are consistently guided in their actions to create a well-structured plot [10]. Despite being emergent narrative systems, none of these examples can be considered examples of interactive storytelling because they do not allow a user to change the path of the story at run-time (\textit{Novel Writer} and \textit{TailSpin} only accepted user input for the initial storyworld state and \textit{Virtual Storyteller} allows for no user input whatsoever, relying solely on its encapsulated virtual director to direct the plot).

Succeeding these establishing approaches to digital storytelling came a greater focus on emergent storytelling and a move towards non-branching, character-focused systems. 
[Talk about coining of the term "emergent narrative"]

[Talk about the success of 


\subsection{Commerical Successes}
Sims, Dwarf Fortress and Rimworld
Facade was an interactive story created by Mateas and Stern in 2005 [CITATION]

\subsection{Emergent Storytelling Approaches in Research}
Emergent Narrative introduced by Aylett in 1999 [6]
FearNot! [CITATION]

\section{A New Approach}

\section{Project Progress}

\newpage
\addcontentsline{toc}{section}{References}

\begin{thebibliography}{20}

\bibitem{Bostan and Marsh}
Bostan B., \& Marsh T. (2012). 
"Fundamentals Of Interactive Storytelling". 
\textit{Academic Journal of Information Technology}, \textbf{3}(8), pp. 20-42

\bibitem{Riedl}
Riedl M. O. (2010). 
"A Comparison of Interactive Narrative System Approaches using Human Improvisational Actors".
\textit{Georgia Institute of Technology, College of Computing}

\bibitem{Authoring Emergent Games}
Louchart S., Aylett R., Kriegel M., Dias J., Figueiredo R., \& Paiva A. (2008).
"Authoring Emergent Narrative-Based Games".
\textit{Journal of Game Development}, \textbf{3}(1), pp. 19–37

\bibitem{Solving the Narrative Paradox in VEs}
Louchart S., \& Aylett R. (2003).
"Solving the Narrative Paradox in VEs - Lessons from RPGs"
\textit{In: Rist T., Aylett R.S., Ballin D., Rickel J. (eds) Intelligent Virtual Agents. IVA 2003. Lecture Notes in Computer Science, Vol. 2792}, pp. 244-248

\bibitem{EmergentNarrative}
Aylett R. (2000).
"Emergent Narrative, Social Immersion and "Storification"".
\textit{Proceedings Narrative and Learning Environments Conference}, Edinburgh, Scotland

\bibitem{Towards Emergent Narrative}
Aylett R. (1999).
"Narrative in Virtual Environment - Towards Emergent Narrative"
\textit{In: Proceedings of the AAAI Fall Symposium on Narrative Intelligence, 1999}

\bibitem{Massively Collaborative Authoring}
Kriegel M., \& Aylett R. (2008).
"Emergent Narrative As A Novel Framework For Massively Collaborative Authoring".
\textit{In: Prendinger H., Lester J., Ishizuka M. (eds) Intelligent Virtual Agents. IVA 2008. Lecture Notes in Computer Science, vol 5208}

\bibitem{Novel Writer}
Klein S., Aeschlimann J.F., Balsiger D.F., Converse S.L., Court C., Foster M., Lao R., Oakley J.D., \& Smith J. (1973).
"Automatic Novel Writing: A Status Report"
\textit{University of Wisconsin}

\bibitem{Tale-Spin}
Meehan, J.R. (1977).
"Tail-Spin, an Interactive Program that Writes Stories"
\textit{In: Proceedings of the Fifth International Joint Conference on Artificial Intelligence (IJCAI)}

\bibitem{The Virtual Storyteller}
Theune, M., Faas S., Nijholt, A., Heylen, D. (2003)
"The Virtual Storyteller: Story Creation by Intelligent Agents"
\textit{In: Proceedings of TIDSE 2003: Technologies for Interactive Digital Storytelling and Entertainment}

\end{thebibliography}
 
\end{document}
